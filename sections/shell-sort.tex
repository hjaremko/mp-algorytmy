\documentclass[advanced-sorts.tex]{subfiles}

\begin{document}

    \begin{minted}[mathescape,
               linenos,
               numbersep=5pt,
               gobble=2,
               frame=lines,
               framesep=2mm]{java}
    void shellSort(int[] arr, int n) {
        int h = 1; //przyrost

        while (h <= n / 3) {
            h = h * 3 + 1;
        }

        while (h > 0) { //zmniejszamy h, aż do momentu h == 1
            for (int k = h; k < n; k++) {
                int tmp = arr[k];
                int j = k;

                while (j >= h && arr[j - h] >= tmp) {
                    arr[j] = arr[j - h];
                    j -= h;
                }

                arr[j] = tmp;
            }

            h = (h - 1) / 3;
        }
    }
    \end{minted}
    % \pagebreak

    \textbf{ZŁOŻONOŚĆ}
    \begin{itemize}
        \item \textbf{Pesymistyczna:} ~~$O(n (\log_2 n)^2)$
        \item \textbf{Pamięciowa:} ~~$\Theta(1)$
    \end{itemize}

    \textbf{OPIS METODY}
    \begin{itemize}
        \item W tej metodzie stosuje się wielokrotnie sortowanie przez
            wstawianie dla elementów odległych od siebie nazywaną
            \textit{przyrostem}, który maleje od pewnej wartości by w końcu
            przyjąć wartość \texttt{1}.
    \end{itemize}

\end{document}
