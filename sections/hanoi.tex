\documentclass[../algorytmy.tex]{subfiles}

\begin{document}

    \textbf{OPIS METODY}\\

    Niech na wieży źródłowej - \texttt{A} znajduje się $n$ krążków, chcemy
    przenieść wszystkie krążki z wieży \texttt{A} na wieżę docelową -
    \texttt{B}, przy czym dostępna jest wieża pomocnicza \texttt{C}.

    \begin{enumerate}
        \item Przenieś poddrzewo składające się z $n-1$ krążków z wieży
            \texttt{A} na wieżę \texttt{C}.
        \item Przenieś ostatni (największy krążek) z \texttt{A} na wieżę
            docelową \texttt{B}.
        \item Przenieś poddrzewo z wieży \texttt{C} na \texttt{B}.
    \end{enumerate}

    \textbf{PSEUDOKOD}
    \begin{minted}[mathescape,
               linenos,
               numbersep=5pt,
               gobble=2,
               frame=lines,
               framesep=2mm]{java}
    void Towers(n, A, B, C) {
        if (n == 0)
            return;

        Towers(n - 1, A, C, B);
        A -> B;
        Towers(n - 1, C, B, A);
    }
    \end{minted}

    \textbf{ZŁOŻONOŚĆ}
    \begin{itemize}
        \item \textbf{Pesymistyczna:} ~~$\Theta(2^n)$
    \end{itemize}

    % \pagebreak

\end{document}
