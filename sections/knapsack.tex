\documentclass[../algorytmy.tex]{subfiles}

\begin{document}

    \textbf{ALGORYTM}
    \begin{enumerate}
        \item Jeśli w jakimkolwiek momencie realizacji procesu suma wag
            wybranych elementów będzie równa  wadze docelowej, należy zakończyć
            działanie (sukces).
        \item Początkowo wybierany jest pierwszy element. Po wybraniu wyznaczamy
            nową wagę docelową, jako różnicę dotychczasowej wagi docelowej i
            wagi pierwszego wybranego elementu. Jeśli suma wag wybranych
            elementów nie będzie równa  wadze docelowej, należy wybrać następny
            element.
        \item Kolejno należy wypróbować wszystkie dostępne kombinacje
            pozostałych elementów. Należy jednak zauważyć, że w rzeczywistości
            wcale nie trzeba sprawdzać wszystkich kombinacji gdyż sumowanie
            można zakończyć w momencie, gdy sumaryczna waga wybranych elementów
            przekracza wagę docelową.
        \item Jeśli nie uda się odnaleźć kombinacji elementów o zadanej wadze,
            to należy odrzucić \underline{pierwszy} element i rozpocząć cały
            proces od początku, wybierając element kolejny.
        \item W podobny sposób należy rozpocząć cały proces sprawdzania,
            wybierając na początku trzeci, czwarty oraz kolejne elementy, aż do
            momentu przeanalizowania całego zbioru dostępnych elementów. Jeśli
            sprawdzenie wszystkich możliwości nie zakończy się sukcesem, będzie
            to oznaczać, że poszukiwane rozwiązanie \underline{nie istnieje}.
    \end{enumerate}

    Rozwiązująca ten problem metoda rekurencyjna mogłaby wybrać pierwszy element
    ze zbioru dostępnych elementów, a następnie, jeśli waga jest mniejsza od
    sumarycznej wagi docelowej, wywołać samą siebie, by sprawdzić sumę wag
    pozostałych dostępnych elementów.

    \pagebreak

\end{document}
