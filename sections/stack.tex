\documentclass[algorytmy.tex]{subfiles}

\begin{document}
    \subsection{Przy użyciu listy wiązanej}
    Lista wiązana \underline{bez nagłówka} jest bardzo efektywną realizacją
    stosu. Zmienna \texttt{top} wskazuje wierzchołek stosu, który jest na
    początku listy, zatem - operacje wstawiania \texttt{push(x)} i usuwania
    \texttt{pop()} polegają na wstawianiu lub usuwaniu pierwszego elementu
    listy wiązanej.

    \subsection{Dostęp do największego elementu w czasie O(1)}
    W każdej komórce przechowujemy aktualną największą wartość.


\end{document}
