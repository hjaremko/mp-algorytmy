\documentclass[algorytmy.tex]{subfiles}

\begin{document}

% \subsection{Terminologia}
% \begin{itemize}
%     \item \textbf{Stopień wierzchołka} \texttt{v} \textit{(deg v)}
%         \begin{itemize}
%             \item Liczba krawędzi grafu incydentnych do wierzchołka. \\
%             Jest on równy sumie liczb wszystkich krawędzi wchodzących i
%             wychodzących. Każdą pętlę liczy się jak dwie krawędzie.\\

%             W grafach skierowanych można wyróżnić stopień wchodzący i
%             wychodzący.
%         \end{itemize}
%     \item \textbf{Stopień grafu}
%         \begin{itemize}
%             \item Maksymalny stopień wierchołka w grafie.
%         \end{itemize}

% \end{itemize}

% \subsection{Reprezentacja}
%     \subsubsection{Macierz sąsiedztwa}
%         \paragraph{Implementacja}
%     \subsubsection{Listy sąsiedztwa}
%         \paragraph{Implementacja}

\subsection{Przeglądanie wszerz (BFS)}

    % \textbf{ZAŁOŻENIA}
    % \begin{itemize}
    %     \item Wszystkie wierzchołki są osiągalne z wierzchołka \texttt{s}.
    %     % \item Dany graf $G = (V, E)$, ~~$L(u)$ - lista węzłów przyległa do u
    % \end{itemize}

    \textbf{OPIS METODY}\\

    Startujemy z \texttt{s}, chcemy dotrzeć do wszystkich wierzchołków
    następującej kolejności: najpierw wierzchołki odległe od startowego o $1$,
    następnie o $2$, następnie o $3$, itd.
    % \begin{itemize}
    %     \item najpierw wierzchołki odległe od startowego o $1$
    %     \item następnie wierzchołki odległe o $2$
    %     \item następnie wierzchołki odległe o $3$, itd
    % \end{itemize}

    Podstawa realizacji - \textbf{kolejka}, która przechowuje węzły do których
    już dotarliśmy, ale dla których jeszcze nie badaliśmy możliwych wyjść
    prowadzących dalej. \\

    \textbf{ZŁOŻONOŚĆ:} ~~$\Theta(n + m)$
    % \begin{itemize}
        % \item \textbf{Czasowa:} ~~$\Theta(n + m)$
        % \item \textbf{Pamięciowa:} ~~$O(n + m)$
    % \end{itemize}

    % \pagebreak

    \textbf{PSEUDOKOD}
    \begin{minted}[mathescape,
               linenos,
               numbersep=5pt,
               gobble=2,
               frame=lines,
               fontsize=\footnotesize,
               framesep=2mm]{java}
    void breadthFirstSearch(Graph G, Node s) {
        G.vertexList[s].visited = true; // oznacz jako odwiedzony
        G.displayVertex(s);               // wyświetl

        Queue Q = new Queue();
        Q.insert(s); // wstaw do kolejki
        int v = 0;

        while (!Q.isEmpty()) { // do opróżnienia kolejki
            int u = Q.first();

            // dopóki nie ma odwiedzonych sąsiadów
            // do v pobierz kolejnego sąsiada u
            while ((v = G.getAdjUnvisitedVertex(u)) != -1) {
                G.vertexList[v].visited = true; // oznacz v jako odwiedzony
                G.displayVertex(v);
                Q.insert(v);
            }

            Q.delete(); // usuń u
        }
    }
    // zwraca nie odwiedzony wierzchołek przyległy do v
    int getAdjUnvisitedVertex(int v) {
        for (int j = 0; j < nVerts; j++)
            if (adjMat[v][j] == 1 && vertexList[j].visited == false)
                return j;
        return -1;
    }
    \end{minted}

\pagebreak

\subsection{Przeglądanie w głąb (DFS)}

    % \textbf{ZAŁOŻENIA}
    % \begin{itemize}
    %     \item Wszystkie wierzchołki są osiągalne z wierzchołka \texttt{s}.
    %     \item Dany graf $G = (V, E)$, $L(u)$ - lista węzłów przyległa do u
    % \end{itemize}

    \textbf{OPIS METODY}\\

    Idź do nowych wierzchołków najdalej jak się da, jeśli dalej nie można to
    wycofaj się do poprzedniego i próbuj inną krawędzią.

    Podstawa realizacji - \textbf{stos}, który można zrealizować niejawnie,
    za pomocą rekurencji.\\

    \textbf{ZŁOŻONOŚĆ:} ~~$\Theta(n + m)$ \\
    % \begin{itemize}
        % \item \textbf{Czasowa:} ~~$\Theta(n + m)$
        % \item \textbf{Pamięciowa:} ~~$O(n + m)$
    % \end{itemize}

    \textbf{PSEUDOKOD} \textit{(iteracyjnie)}
    \begin{minted}[mathescape,
               linenos,
               numbersep=5pt,
               gobble=2,
               frame=lines,
               fontsize=\footnotesize,
               framesep=2mm]{java}
    void depthFirstSearch(Graph G) { //rozpocznij od węzła 0
        G.vertexList[0].visited = true;
        G.displayVertex(0);

        Stack S = new Stack();
        S.push(0);

        while (!S.isEmpty()) {
            // pobierz nie odwiedzony węzeł przyległy do
            // szczytowego elementu stosu (u)
            int v = G.getAdjUnvisitedVertex(S.top());

            if (v == -1)
                S.pop();
            else { // jeżeli istnieje oznacz v
                G.vertexList[v].visited = true;
                G.displayVertex(v);
                S.push(v);
            }
        }
    }
    \end{minted}

    \textbf{PSEUDOKOD} \textit{(rekurencyjnie)}
    \begin{minted}[mathescape,
               linenos,
               numbersep=5pt,
               gobble=2,
               frame=lines,
               fontsize=\footnotesize,
               framesep=2mm]{java}
    void depthFirstSearch(Graph G, int v) {
        G.vertexList[v] = true;
        G.displayVertex(v);
        int i = 0;

        while ((i = G.getAdjUnvisitedVertex(v)) != -1) {
            dfs(G, i);
        }
    }
    \end{minted}

\subsection{Przeglądanie w głąb (DFS) (Cormen)}
    \textbf{PSEUDOKOD}% \textit{(Cormen)}
    \begin{minted}[mathescape,
               linenos,
               numbersep=5pt,
               gobble=2,
               frame=lines,
               fontsize=\footnotesize,
               framesep=2mm]{java}
    DFS(G) {
        for each u in V do { //inicjalizacja
            color[u] = white;
            P[u] = null; //poprzednik w drzewe
        }

        time = 0; //globalna

        for each u in V do {
            if (color[u] == white)
                DFS-Visit(u);
        }
    }

    DFS-Visit(u) {
        color[u] = grey;
        d[u] = ++time;

        //badaj (u, v)
        for each v in L[u] do {
            if (color[v] == white) {
                P[v] = u;
                DFS-Visit(v);
            }
        }

        color[u] = black;
        f[u] = ++time;
    }
    \end{minted}


\begin{itemize}
    \item \texttt{L[]} - lista sąsiedztwa
    \item \texttt{P[]} - poprzednik w drzewie
    \item \texttt{d[]} - czas odwiedzenia
    \item \texttt{f[]} - czas przetworzenia
\end{itemize}

\pagebreak

\subsection{Minimalne drzewo rozpinające (MST)}

    \textbf{OPIS METODY} \\\\
    \textbf{Minimalne drzewo rozpinające} to podgraf o \underline{najmniejszej}
    \underline{liczbie krawędzi}
    \underline{wymaganej} \underline{do połączenia w zadanym grafie}
    (spójnym/silnie spójnym). Dla danego grafu spójnego istnieje wiele różnych
    minimalnych drzew rozpinających.\\

    \textbf{PSEUDOKOD} \textit{(iteracyjnie)}
    \begin{minted}[mathescape,
               linenos,
               numbersep=5pt,
               gobble=2,
               frame=lines,
               % fontsize=\footnotesize,
               framesep=2mm]{java}
    void minimalSpanningTree(Graph G) {
        G.vertexList[0].visited = true;

        Stack S = new Stack();
        S.push(0);

        while (!S.isEmpty()) {
            int u = S.top();
            // pobierz nie odwiedzony węzeł przyległy do
            // szczytowego elementu stosu (u)
            int v = G.getAdjUnvisitedVertex(u);

            if (v == -1)
                S.pop();
            else { //jeżeli istnieje
                G.vertexList[v].visited = true;
                S.push(v);

                // wyświelt krawędź od u do v
                G.displayVertex(u);
                G.displayVertex(v); //print(',');
            }
        }
    }
    \end{minted}

\pagebreak
\subsection{Test acykliczności grafu skierowanego}

\textbf{OPIS METODY}
\begin{itemize}
    \item Graf zorientowany zawiera cykl wtedy i tylko wtedy gdy w dowolnym
        drzewie \textbf{DFS} istnieje krawędź \textbf{wsteczna}.
    \item To znaczy, gdy podczas przeglądania grafu metodą \textbf{DFS}
        odwiedzimy wierzchołek pokolorowany na \textbf{szaro} to graf zawiera
        cykl.
\end{itemize}

    \textbf{PSEUDOKOD} \textit{(Cormen)}
    \begin{minted}[mathescape,
               linenos,
               numbersep=5pt,
               gobble=2,
               frame=lines,
               fontsize=\footnotesize,
               framesep=2mm]{java}
    DFS(G) {
        for each u in V do { //inicjalizacja
            color[u] = white;
        }

        for each u in V do {
            if (color[u] == white)
                if (DFS-Visit(u) == true)
                    return true;
        }

        return false;
    }

    DFS-Visit(u) {
        color[u] = grey;

        //badaj (u, v)
        for each v in L[u] do {
            if (color[v] == white) {
                if (DFS-Visit(v) == true)
                    return true;
            }
            else if (color[v] == grey) {
                return true;
            }
        }

        color[u] = black;
        return false;
    }
    \end{minted}

\pagebreak


\subsection{Wyznaczanie spójnych składowych grafu nieskierowanego}
\begin{itemize}
    \item Należy obliczyć \texttt{ss[u] = k} - numer spójnej składowej
    zawierającej wierzchołek \texttt{u}, dla każdego $u \in V$ wierzchołki w tej
    samej składowej dostaną jeden numer \texttt{ss}.
    \item Zmienną globalną \texttt{k}, zerowaną na początku, zwiększamy o 1 przy
    każdym wywołaniu funkcji \texttt{DFS-Visit} z głównej pętli w algorytmie
    DFS.
    \item Na początku funkcji \texttt{DFS-Visit} wykonujemy instrukcję
    \texttt{ss[u] = k}.
\end{itemize}

Do obliczenia \texttt{ss[]} w algorytmie \textbf{DFS} można opuścić:
\begin{itemize}
    \item trzeci kolor (wystarczą dwa)
    \item poprzednik w drzewie, \texttt{P[]}
    \item tablice \texttt{d[]}, \texttt{f[]}.
\end{itemize}

Oba powyższe problemy można rozwiązać również wykorzystując algorytm
\textbf{BFS}.

    \textbf{PSEUDOKOD} \textit{(Cormen)}
    \begin{minted}[mathescape,
               linenos,
               numbersep=5pt,
               gobble=2,
               frame=lines,
               fontsize=\footnotesize,
               framesep=2mm]{java}
    DFS(G) {
        for each u in V do //inicjalizacja
            color[u] = white;

        k = 0; //globalna

        for each u in V do {
            if (color[u] == white) {
                k++;
                DFS-Visit(u);
            }
        }
    }

    DFS-Visit(u) {
        ss[u] = k;
        color[u] = grey;

        for each v in L[u] do {
            if (color[v] == white)
                DFS-Visit(v);
        }
    }
    \end{minted}

\pagebreak

\subsection{Sortowanie topologiczne}
    \textbf{OPIS METODY}
    \begin{itemize}
        \item Zastosuj algorytm \texttt{DFS(G)}, wpisując wierzchołek \texttt{u}
              na początek listy w momencie jego kolorowania na \textbf{czarno}.
    \end{itemize}

    \textbf{PSEUDOKOD} \textit{(Cormen)}
    \begin{minted}[mathescape,
               linenos,
               numbersep=5pt,
               gobble=2,
               frame=lines,
               % fontsize=\footnotesize,
               framesep=2mm]{java}
    DFS(G) {
        for each u in V do //inicjalizacja
            color[u] = white;

        for each u in V do {
            if (color[u] == white) {
                if (DFS-Visit(u) == false)
                    return;
            }
        }
    }

    DFS-Visit(u) {
        color[u] = grey;

        for each v in L[u] do {
            if (color[v] == white) {
                if (DFS-Visit(v) == false)
                    return false;
            }
            else if (color[u] == grey)
            //graf zawiera cykl -> nie można posortować!
                return false;
        }

        color[u] = black;
        print(u);
        return true;
    }
    \end{minted}

\pagebreak

\subsection{Średnica grafu nieskierowanego}
    \textbf{OPIS METODY}
    \begin{itemize}
        \item Obliczamy największą odległość dla każdego wierzchołka stosując
            algorytm \textbf{BFS}.
    \end{itemize}

    \textbf{PSEUDOKOD}
    \begin{minted}[mathescape,
               linenos,
               numbersep=5pt,
               gobble=2,
               frame=lines,
               fontsize=\footnotesize,
               framesep=2mm]{java}

    int bfs(Graph G, Vertex s) {
        for each u in G.V {
            color[u] = white;
            dist[u] = Integer.MAX_VALUE; //infinity
        }

        color[s] = grey;
        dist[s] = 0;
        Q.push(s);

        while (!Q.isEmpty()) {
            int u = Q.pop();

            for each v in L[u] {
                if (color[v] == white) {
                    color[v] = grey;
                    dist[v]++;
                    Q.push(v);
                }
            }
        }

        return maxValue(dist);
    }

    int diameter() {
        int result = 0;

        for (int i = 0; i < n; i++) {
            int d = bfs(i);

            if (d > result)
                d = result;
        }

        return result;
    }
    \end{minted}

\pagebreak

\end{document}
