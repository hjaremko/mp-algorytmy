\documentclass[advanced-sorts.tex]{subfiles}

\begin{document}

    \textbf{REKURENCYJNIE}
    \begin{minted}[mathescape,
               linenos,
               numbersep=5pt,
               gobble=2,
               frame=lines,
               framesep=2mm]{java}
    void mergeSort(int[] arr, int left, int right) {
        if (left < right) {
            int middle = (left + right) / 2;

            mergeSort(arr, left, middle);
            mergeSort(arr, middle + 1, right);
            merge(arr, left, middle, right);
        }
    }

    void merge(int[] arr, int left, int mid, int right) {
        int[] temp = new int[arr.length];

        for (int i = left; i <= right; i++)
            temp[i] = arr[i];

        int i = left;
        int j = mid + 1;
        int k = left;

        while (i <= mid && j <= right) {
            if ( temp[i] <= temp[j] )
                arr[k++] = temp[i++];
            else
                arr[k++] = temp[j++];
        }

        while (i <= mid) // && i < n) w iteracyjnym
            arr[k++] = temp[i++];
        while (j <= right) // nie trzeba w iteracyjnym
            arr[k++] = temp[j++];

    }
    \end{minted}
    \pagebreak

    \textbf{ITERACYJNIE}
    \begin{minted}[mathescape,
               linenos,
               numbersep=5pt,
               gobble=2,
               frame=lines,
               framesep=2mm]{java}
    void mergeSort(int[] arr, int left, int right) {
        for (int size = 1; size <= n - 1; size = 2 * size) {
            for (int left = 0; left < n - 1; left += 2 * size) {
                int mid = min(left + size - 1, arr.length - 1);
                right = min(left + 2 * size - 1, arr.length - 1);

                merge(left, mid, right);
            }
        }
    }
    \end{minted}

    \textbf{ZŁOŻONOŚĆ}
    \begin{itemize}
        \item \textbf{Pesymistyczna:} ~~$\Theta(n \log_2 n)$
        \item \textbf{Pamięciowa:} ~~$\Theta(n)$
    \end{itemize}

    \textbf{ZALETY}
    \begin{itemize}
        \item \textbf{Stabilna.}
    \end{itemize}

    \textbf{WADY}
    \begin{itemize}
        \item Pamięć robocza rozmiaru $\Theta(n)$.
        \item Czasochłonne przepisywanie elementów.
    \end{itemize}

    % \textbf{MOŻLIWE USPRAWNIENIA}
    % \begin{itemize}
    %     \item .
    % \end{itemize}

\end{document}
