\documentclass[algorytmy.tex]{subfiles}

\begin{document}
    \subsection{Prosta przy użyciu listy wiązanej dwustronnej}
    \subsubsection{Wstawianie}
    \begin{minted}[mathescape,
               linenos,
               numbersep=5pt,
               gobble=2,
               frame=lines,
               % fontsize=\footnotesize,
               framesep=2mm]{java}
    void enqueue(T x) {
        Node newNode = new Node(x);
        rear.next = newNode;
        rear = newNode;
    }
   \end{minted}

    \subsubsection{Usuwanie}
    \begin{minted}[mathescape,
               linenos,
               numbersep=5pt,
               gobble=2,
               frame=lines,
               % fontsize=\footnotesize,
               framesep=2mm]{java}
    T dequeue() {
        T tmp = front.next.info;
        front = front.next;

        return tmp;
    }
   \end{minted}

    \subsection{Priorytetowa przy użyciu uporządkowanej listy wiązanej}

    \pagebreak
    \subsection{Prosta przy użyciu tablicy}
    \begin{minted}[mathescape,
               linenos,
               numbersep=5pt,
               gobble=2,
               frame=lines,
               % fontsize=\footnotesize,
               framesep=2mm]{java}
    class Queue {
        int maxSize;
        long[] elem;
        int front = 0;
        int rear = 0;

        private int addOne(int i) {
            return (i + 1) % maxSize;
        }


        void enqueue(long x) {
            if (isFull())
                error("Queue is full");
            else {
                elem[rear] = x;
                rear = addOne(rear);
            }
        }

        long dequeue() {
            if (isEmpty()) {
                error("Queue is empty");
                return -1;
            }
            else {
                long tmp = elem[front];
                front = addOne(front);
                return tmp;
            }
        }

        boolean isFull() {
            return (addOne(rear) == front);
        }
    }
   \end{minted}

    % \subsection{Priorytetowa przy użyciu tablicy}
    % \subsection{Dwustronna przy użyciu tablicy}

\pagebreak

\end{document}
