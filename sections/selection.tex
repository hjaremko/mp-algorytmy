\documentclass[../algorytmy.tex]{subfiles}

\begin{document}

\subsection{Metoda Hoare}

    \textbf{PSEUDOKOD}
    \begin{minted}[mathescape,
               linenos,
               numbersep=5pt,
               gobble=2,
               frame=lines,
               % fontsize=\footnotesize,
               framesep=2mm]{java}
    Item Select1(S, k) {
    // znajdowanie k-tego (1 < k < n) co do wielkości elementu
    // zbioru S, licząc od najmniejszego
    // (k = 1 - element najmniejszy)
        a = dowolny element zbioru S;
        Przeglądnij zbiór S i wyznacz zbiory:
            S1 = {x in S : x < a};
            S2 = {x in S : x = a};
            S3 = {x in S : x > a};

        if (k <= |S1| ) return Select1(S1, k);
            // szukany element jest k-tym elementem w S1
        if (k <= |S1| + |S2|) return a;
            // szukany jest w S2, czyli równy a
        return Select1(S3, k - |S1| - |S2|);
            // szukaj w S3, numer odpowiednio mniejszy
    }
   \end{minted}

    \textbf{ZŁOŻONOŚĆ}
    \begin{itemize}
        \item \textbf{Czasowa pesymistyczna:} ~~$O(n^2)$ \textit{(jak quicksort)}
        \item \textbf{Czasowa średnia:} ~~$O(n)$
    \end{itemize}

\pagebreak

\subsection{Algorytm magicznych piątek}

    \textbf{PSEUDOKOD}
    \begin{minted}[mathescape,
               linenos,
               numbersep=5pt,
               gobble=2,
               frame=lines,
               % fontsize=\footnotesize,
               framesep=2mm]{java}
    Item Select2(S, k) {
        (1) Jeśli n < p to posortuj i wypisz k-ty element. //baza
        (2) Podziel S na 5-elementowe podzbiory i ewentualnie
            jeden co najwyżej 4-elementowy.
        (3) Posortuj każdy podzbiór oddzielnie.
        (4) Wyznacz nowy zbiór Q = {środkowe elementy z każdego
                                    podzbioru}
        (5) Wyznacz M = Select2(Q, |Q| / 2)
            //rekurancja, mediana median
        (6) Dalej jak w metodzie Hoare:

        Przeglądnij zbiór S i wyznacz zbiory:
            S1 = {x in S : x < M};
            S2 = {x in S : x = M};
            S3 = {x in S : x > M};

        if (k <= |S1| ) return Select2(S1, k);
            // szukany element jest k-tym elementem w S1
        if (k <= |S1| + |S2|) return M;
            // szukany jest w S2, czyli równy M
        return Select2(S3, k - |S1| - |S2|);
            // szukaj w S3, numer odpowiednio mniejszy
    }
   \end{minted}

    \textbf{ZŁOŻONOŚĆ}
    \begin{itemize}
        \item \textbf{Czasowa pesymistyczna:} ~~$O(n)$
    \end{itemize}


\end{document}
