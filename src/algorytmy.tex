% !TEX options=--shell-escape
\documentclass[a5paper, 4pt]{article}
% \usepackage{lmodern}
\usepackage[MeX]{polski}
\usepackage{geometry}
\usepackage{fontspec}
\usepackage{color}
\usepackage{adjustbox}
\usepackage{minted,xcolor}
\usemintedstyle{trac}
% \definecolor{bg}{HTML}{282828} % from https://github.com/kevinsawicki/monokai
% \setminted{bgcolor=bg}
\usepackage{tcolorbox}
\usepackage{amsmath}

\usepackage{subfiles}

\setmainfont[Path = fonts/lato/,
BoldItalicFont=Lato-RegIta,BoldFont=Lato-Reg,ItalicFont=Lato-LigIta]{Lato-Lig}
\setmonofont[Scale = 0.9, Path = fonts/fira/]{FiraCode-Regular}

\newgeometry{tmargin=2cm, bmargin=2cm, lmargin=1.5cm, rmargin=1.5cm}
\setlength{\parindent}{0cm}

% \linespread{0.9}

\newcommand{\horrule}[1]{\rule{\linewidth}{#1}}
\newcommand\tab[1][1cm]{\hspace*{#1}}


\title{
\vskip 2cm
  Metody Programowania}
  % \large Algorytmy}
\author{Hubert Jaremko}
% \date{2018/2019}
\date{\today}
\frenchspacing

\begin{document}
    \maketitle
    \pagebreak
    \tableofcontents
    \pagebreak

    \section{Sortowania proste}
    \subfile{sections/simple-sorts.tex}

    \section{Sortowania zaawansowane}
    \subfile{sections/advanced-sorts.tex}

    \section{Wyszukiwania}
    \subfile{sections/search-algorithms.tex}

    \section{Selekcja}
    \subfile{sections/selection.tex}

    \pagebreak

    \section{Liczba inwersji}
    \subfile{sections/inversions.tex}

    \section{Algorytm Kadane}
    \subfile{sections/kadane.tex}

    \section{Problem plecakowy}
    \subfile{sections/knapsack.tex}

    \section{Wieże Hanoi}
    \subfile{sections/hanoi.tex}

    \section{Eliminacja rekurencji}
    \subfile{sections/recursion.tex}

    % \section{Tablice}
    \subfile{sections/array.tex}

    \section{Listy}
    \subfile{sections/list.tex}

    \section{Stosy}
    \subfile{sections/stack.tex}

    \section{Kolejki}
    \subfile{sections/queue.tex}

    \section{Drzewa}
    \subfile{sections/tree.tex}

    \section{Kopce}
    \subfile{sections/heap.tex}

    \section{Grafy}
    \subfile{sections/graph.tex}

    \section{Grafy ważone}
    \subfile{sections/weighted-graph.tex}

    \newpage
    ~
\end{document}
