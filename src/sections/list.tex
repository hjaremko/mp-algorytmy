\documentclass[../algorytmy.tex]{subfiles}

\begin{document}

    \subsection{Jednokierunkowa}

    % \subsubsection{Budowa}

    % \begin{minted}[mathescape,
    %            linenos,
    %            numbersep=5pt,
    %            gobble=2,
    %            frame=lines,
    %            fontsize=\footnotesize,
    %            framesep=2mm]{java}
    % class List<T> {
    %     class Node {
    %         T data;
    %         Node next = null;
    %     }

    %     Node first = null; //bez głowy
    % }
    % \end{minted}
    %     % // Node head = new Node();

    % \subsubsection{Wyszukiwanie}

    % % \textbf{Z GŁOWĄ}
    % \begin{minted}[mathescape,
    %            linenos,
    %            numbersep=5pt,
    %            gobble=2,
    %            frame=lines,
    %            % fontsize=\footnotesize,
    %            framesep=2mm]{java}
    % Node find(T key) {
    %     Node current = first;

    %     while (current != null && !current.data.equals(key))
    %         current = current.next;

    %     return current;
    % }
    % \end{minted}

    % \textbf{BEZ GŁOWY}

    \subsubsection{Wstawianie za podanym elementem}
    \begin{minted}[mathescape,
               linenos,
               numbersep=5pt,
               gobble=2,
               frame=lines,
               % fontsize=\small,
               framesep=2mm]{java}
    void insertAfter(T elem, Node p) {
        Node newElem = new Node(elem);

        if (p == null) { //wstawianie na poczatek
            newElem.next = first;
            first = newElem;
        }
        else {
            newElem.next = p.next;
            p.next = newElem;
        }
    }
    \end{minted}

    \pagebreak
    \subsubsection{Usuwanie elementu o podanej wartości}
    \begin{minted}[mathescape,
               linenos,
               numbersep=5pt,
               gobble=2,
               frame=lines,
               framesep=2mm]{java}
    void delete(T key) {
        Node curr = first;
        Node prev = null;

        while (curr != null && p.data != key) {
            prev = curr;
            curr = curr.next;
        }

        if (curr != null) {
            if (prev == null) {
                first = p.next;
            }
            else {
                prev.next = p.next;
            }
        }
    }
    \end{minted}

    \subsection{Dwukierunkowa}
    \subsubsection{Usuwanie elementu o podanej wartości}
    \begin{minted}[mathescape,
               linenos,
               numbersep=5pt,
               gobble=2,
               frame=lines,
               framesep=2mm]{java}
    void delete(T key) {
        Node elem = find(key);

        if (curr != null) {
            elem.prev.next = elem.next;
            elem.next.prev = elem.prev;
        }
    }
    \end{minted}

    \subsubsection{Wstawanie elementu za elementem o podanej wartości}
    \begin{minted}[mathescape,
               linenos,
               numbersep=5pt,
               gobble=2,
               frame=lines,
               framesep=2mm]{java}
    void insertAfter(T x, T key) {
        Node elem = find(key);
        Node newNode = new Node(x);

        if (elem == last) {
            newNode.next = null;
            last = newNode;
        }
        else {
            newNode.next = elem.next;
            elem.next.prev = newNode
        }

        newNode.prev = elem;
        elem.next = newNode;
    }
    \end{minted}

    \subsubsection{Wstawanie elementu na początek}
    \begin{minted}[mathescape,
               linenos,
               numbersep=5pt,
               gobble=2,
               frame=lines,
               framesep=2mm]{java}
    void insertFirst(T value) {
        Node newNode = new Node(value);

        if (isEmpty()) {
            last = newNode;
        }
        else {
            first.prev = newElem;
        }

        newNode.next = first;
        first = newElem;
    }
    \end{minted}


    \subsection{Prosta cykliczna}
    \subsubsection{Budowa}
    W ostatniej komórce referencja \texttt{next} jest  adresem  pierwszej
    komórki listy (lub nagłówka, jeśli wersja z nagłówkiem).
    Zamiast testu \texttt{p != null} należy zastosować test \texttt{p != s},
    gdzie \texttt{s} jest referencją komórki, od której zaczęliśmy przegląd
    listy.

    \subsubsection{Usuwanie elementu o podanej wartości}
    \begin{minted}[mathescape,
               linenos,
               numbersep=5pt,
               gobble=2,
               frame=lines,
               framesep=2mm]{java}
    void insertFirst(T value) {
        Node curr = first;

        while (curr.next != first && curr.next.data != value) {
            curr = curr.next;
        }

        if (curr.next != first) {
            curr.next = curr.next.next;
        }
    }
    \end{minted}

% \pagebreak

\end{document}
