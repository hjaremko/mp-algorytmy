\documentclass[search-algorithms.tex]{subfiles}

\begin{document}

    \begin{minted}[mathescape,
               linenos,
               numbersep=5pt,
               gobble=2,
               frame=lines,
               framesep=2mm]{java}
    void binarySearch(int key) {
        int begin = 0;
        int end = n - 1;

        while (begin <= end) {
            int current = (begin + end) / 2;

            if (arr[current] == key) {
                return current;
            }
            else {
                if (arr[current] < key)
                    begin = current + 1;
                else
                    end = current - 1;
            }
        }

        return -1;
    }
    \end{minted}
    % \pagebreak

    \textbf{ZŁOŻONOŚĆ}
    \begin{itemize}
        \item \textbf{Pesymistyczna:} ~~$\Theta(\log_2 n)$ (optymalne)
        \item \textbf{Średnia:} ~~$\Theta(\log_2 n)$
    \end{itemize}

    % \textbf{MOŻLIWE USPRAWNIENIA}
    % \begin{itemize}
    %     \item .
    % \end{itemize}

    % \pagebreak
    \subsection{Wyszukiwanie interpolacyjne}
    \begin{itemize}
        \item Zakładamy \underline{liniowy rozkład wartości elementów.}
    \end{itemize}

    \begin{equation*}
        \frac{curr - low}{upp - low} =
        \frac{x - \mathtt{a[low]}}{\mathtt{a[upp]} - \mathtt{a[low]}}
    \end{equation*}

    \begin{equation*}
        curr = low + (x - \mathtt{a[low]})
        \frac{upp - low}{\mathtt{a[upp]} - \mathtt{a[low]}}
    \end{equation*}

    % \subsection{Wyszukiwanie binarne pierwszego/ostatniego wystąpienia} %TODO


\end{document}
