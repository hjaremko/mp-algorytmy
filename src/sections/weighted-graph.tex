\documentclass[algorytmy.tex]{subfiles}

\begin{document}

% \subsection{Definicja}
\subsection{Najkrótsze ścieżki przy ustalonym źródle}
    \textbf{OGÓLNA METODA ROZWIĄZYWANIA}
    \begin{itemize}
        \item \texttt{D[v]} - wyliczone dotychczas najlepsze oszacowanie od góry
            dla \texttt{d(s, v)}, początkowo równe $+\infty$.
        \item \texttt{P[v]} - poprzednik wierzchołka \texttt{v}, początkowo
            równy \texttt{null}.

        \begin{minted}[mathescape,
               linenos,
               numbersep=5pt,
               gobble=2,
               frame=lines,
               % fontsize=\footnotesize,
               framesep=2mm]{java}
    Init() {
        for each v in V {
            D[v] = Integer.MAX_VALUE; //+infinity
            P[v] = null;
        }
    }
        \end{minted}

    \item dla wszystkich krawędzi, w odpowiedniej kolejności wykonaj odpowiednią
        liczbę razy procedurę relaksacji:

        \begin{minted}[mathescape,
               linenos,
               numbersep=5pt,
               gobble=2,
               frame=lines,
               % fontsize=\footnotesize,
               framesep=2mm]{java}
    Relax(Vertex u, v) {
        if (D[v] > D[u] + w[u, v]) {
            D[v] = D[u] + w[u, v]; //poprawiono oszacowanie
            P[v] = u; //nowy poprzednik na ścieżce
        }
    }
    \end{minted}
    \end{itemize}

    \textbf{ZŁOŻONOŚĆ}
    \begin{itemize}
        \item Inicjalizacja: $\Theta(n)$
        \item \texttt{Relax}: $\Theta(1)$
    \end{itemize}

    \pagebreak

    \subsubsection{Algorytm Bellman-Ford (B-F)}

    \textbf{OPIS METODY}
    \begin{itemize}
        \item Ponieważ nie ma cykli ujemnych, powtarzanie jakiegokolwiek cyklu
            na ścieżce jej nie skraca, zatem najkrótsze ścieżki są elementarne
            i mają nie więcej niż $n-1$ krawędzi.
        \item Zatem jeśli pierwszy raz wykonamy procedurę \texttt{Relax} dla
            wszystkich krawędzi, to otrzymamy najkrótsze odległości po ścieżkach
            1 - krawędziowych. Powtórzywszy to, otrzymamy najkrótsze odległości po
            ścieżkach 2 - krawędziowych, itd.
    \end{itemize}

    \textbf{ZŁOŻONOŚĆ}
    \begin{itemize}
        \item $\Theta(n \cdot m )$ - gdy graf reprezentowany przez listę
            sąsiedztwa
        \item $\Theta(n^3)$ - gdy graf reprezentowany przez macierz sąsiedztwa
            \textit{(szukając krawędzi z wierzchołka trzeba przeglądać cały
            wiersz tablicy)}.
    \end{itemize}

    \begin{minted}[mathescape,
               linenos,
               numbersep=5pt,
               gobble=2,
               frame=lines,
               % fontsize=\footnotesize,
               framesep=2mm]{java}
    BellmanFord(Graph G = (V, E)) {
        Init();
        D[s] = 0;
        for (i = 1; i < n; i++)
            for each (u, v) in E
                Relax(u, v);
    }
    \end{minted}

    \textbf{SPRAWDZENIE CZY GRAF ZAWIERA UJEMNE CYKLE}
    \begin{itemize}
        \item Zauważmy, że jeśli ma taki cykl, to po zakończeniu B-F, jeszcze
            jedna iteracja pętli \texttt{for each} będzie ‘poprawiać’
            oszacowanie \texttt{D[v]} dla pewnego wierzchołka \texttt{v}.
            Nie powiększa to złożoności algorytmu.
    \end{itemize}

    \begin{minted}[mathescape,
               linenos,
               numbersep=5pt,
               gobble=2,
               frame=lines,
               % fontsize=\footnotesize,
               framesep=2mm]{java}
    boolean Test() {
        for each (u, v) in E
            if (D[v] > D[u] + w[u, v])
                return false; // graf zawiera ujemny cykl
        return true; // graf OK
    }
    \end{minted}

    \pagebreak

    \subsubsection{Algorytm Dijkstra (DIJ 1959)}

    \textbf{OPIS METODY}
    \begin{itemize}
        \item Zakładamy, że wagi krawędzi są \textbf{nieujemne}.
    \end{itemize}
    \begin{enumerate}
        \item W zbiorze \textit{(kolejce)} \texttt{Q} przechowuj wierzchołki
            \texttt{v}, dla których \texttt{D[v]} być może nie ma ostatecznej
            wartości, początkowo \texttt{Q = V} i ustaw \texttt{D[s] = 0}.
        \item W kolejnych krokach:
            \begin{enumerate}
                \item Znajduj $u \in Q$ taki, że \texttt{D[u]} jest
                    \textbf{najmniejsze}.

                \item Usuń \texttt{u} z \texttt{Q} i popraw pozostałym $v \in Q$
                    ich \texttt{D[v]} względem \texttt{u}.
            \end{enumerate}
    \end{enumerate}

    \begin{minted}[mathescape,
               linenos,
               numbersep=5pt,
               gobble=2,
               frame=lines,
               % fontsize=\footnotesize,
               framesep=2mm]{java}
    Dijkstra(G = (V, E)) {
        Init();
        D[s] = 0;
        Q = V; //kolejka wierzchołków

        while (Q != empty) {
            u = DelMin(Q);
            for each v in L[u] // lista następników u
                Relax(u, v);
        }
    }
    \end{minted}

    \textbf{ZŁOŻONOŚĆ:}
    \begin{enumerate}
        \item \texttt{Q} \textbf{jest zwykłą listą:} $\Theta(n^2)$
            \begin{itemize}
                \item wyszukanie i usunięcie minimum: $\Theta(n)$
                \item przeglądnięcie listy następników i wykonanie
                      \texttt{Relax}: $\Theta(n)$
                \item oba powyższe - wykonywane w pętli zwenętrznej $n$ razy
            \end{itemize}

        \item \texttt{Q} \textbf{jest min-kopcem, a graf jest
            reprezentowany przez listy sąsiedztwa:} $\Theta(m \log n)$
            \begin{itemize}
                \item $n$ razy \texttt{DelMin()}: $\Theta(n \log n)$
                \item $m$ raz \texttt{Relax()}: $\Theta(m \log n)$
                      \textit{(każda krawędz raz, koszt przesiewania log n)}
            \end{itemize}
    \end{enumerate}

\pagebreak

\subsection{Najkrótsze ścieżki między wszystkimi wierzchołkami}
    \subsubsection{Algorytm Floyda}

    \begin{minted}[mathescape,
               linenos,
               numbersep=5pt,
               gobble=2,
               frame=lines,
               % fontsize=\footnotesize,
               framesep=2mm]{java}
    Floyd(G = (V, E)) { //O(n^3)
        // inicjalizacja: ścieżki 1-krawędziowe
        for (i = 1; i <= n; i++)
            for (j = 1; j <= n; j++)
                D[i, j] = w[i, j];

        for (k = 1; k <= n; k++)
            for (i = 1; i <= n; i++)
                for (j = 1; j <= n; j++) {
                    t = D[i, k] + D[k, j];
                    if (t < D[i, j]) {
                        D[i, j] = t;
                        P[i, j] = k;
                    }
                }
    }
    \end{minted}

    % \textbf{ZŁOŻONOŚĆ:} $\Theta(n^3)$

\subsection{Przechodnie domknięcie grafu}
    \subsubsection{Algorytm Warshalla}

    \begin{minted}[mathescape,
               linenos,
               numbersep=5pt,
               gobble=2,
               frame=lines,
               fontsize=\footnotesize,
               framesep=2mm]{java}
    Warshall(G = (V, E)) {
        // inicjalizacja
        for (i = 1; i <= n; i++)
            for (j = 1; j <= n; j++)
                if ((i, j) in E)
                    D[i, j] = 1;
                else
                    D[i, j] = 0;

        for (k = 1; k <= n; k++)
            for (i = 1; i <= n; i++)
                for (j = 1; j <= n; j++)
                    if (D[i, j] == 0)
                        D[i, j] = D[i, k] && D[k, j];
    }
    \end{minted}

% \subsection{Problemy grafowe NP-zupełne}
%     \subsubsection{Problem komiwojażera}
%     \subsubsection{Kolorowanie grafów}

\end{document}
