\documentclass[advanced-sorts.tex]{subfiles}

\begin{document}

    \textbf{REKURENCYJNIE}
    \begin{minted}[mathescape,
               linenos,
               numbersep=5pt,
               gobble=2,
               frame=lines,
               % fontsize=\footnotesize,
               framesep=2mm]{java}
    void quickSort(int[] arr, int left, int right) {
        if (left < right) {
            int pivot = partition(arr, left, right);

            quickSort(left, pivot - 1);
            quickSort(pivot + 1, right);
        }
    }
    \end{minted}

    \textbf{ITERACYJNIE}
    \begin{minted}[mathescape,
               linenos,
               numbersep=5pt,
               gobble=2,
               frame=lines,
               % fontsize=\footnotesize,
               framesep=2mm]{java}
    void quickSort(int[] arr, int left, int right) {
        while (left < right || !stack.isEmpty()) {
            if (left < right) {
                int pivot = partition(arr, left, right);

                //na stos prawe podzadanie
                stack.push(right);

                right = pivot - 1;
            }
            else {
                left = right + 2;
                right = stack.pop();
            }
        }
    }
    \end{minted}

    \pagebreak

    \textbf{HOARE}
    \begin{minted}[mathescape,
               linenos,
               numbersep=5pt,
               gobble=2,
               frame=lines,
               framesep=2mm]{java}

    int partition(int[] arr, int left, int right) {
        int i = left - 1;
        int j = right;
        int x = arr[right]; //ostatni elemnt jest dzielący

        while (true) {
            while (arr[++i] < x);
            while (j > left && arr[--j] > x);

            if (i >= j)
                break;
            else
                swap(i, j);
        }

        swap(i, right);
        return i;
    }
    \end{minted}

    \textbf{LOMUTO}
    \begin{minted}[mathescape,
               linenos,
               numbersep=5pt,
               gobble=2,
               frame=lines,
               framesep=2mm]{java}
    int partition(int[] arr, int left, int right) {
        int i = left - 1;
        int x = arr[right]; //ostatni element jest dzielący

        for (int j = left; j < right; j++) {
            if (arr[j] <= x) {
                swap(++i, j);
            }
        }

        swap(i + 1, right);
        return i + 1;
    }
    \end{minted}

    \pagebreak

    \textbf{ZŁOŻONOŚĆ}
    \begin{itemize}
        \item \textbf{Optymistyczna:} ~~$\Theta(n \log_2 n)$
        \item \textbf{Średnia:} ~~$\Theta(n \log_2 n)$
        \item \textbf{Pesymistyczna:} ~~$\Theta(n^2)$
        \item \textbf{Pamięciowa:} Konieczność stosu:~~$\Theta(n)$,
            usprawniony~~$\Theta(\log_2 n)$
    \end{itemize}

    \textbf{ZALETY}
    \begin{itemize}
        \item Jest to w ogólnym przypadku tablicy najszybszy algorytm
            wykorzystujący operację porównywania elementów.
    \end{itemize}

    \textbf{WADY}
    \begin{itemize}
        \item \textbf{Niestabilność.}
        \item Koszt $\Theta(n^2)$ w przypadku pesymistycznym.
    \end{itemize}

    \textbf{MOŻLIWE USPRAWNIENIA}
    \begin{itemize}
        \item W przypadku małych podzadań (np. \texttt{n <= 20}) sortuj prostą
            metodą (zalecana metoda przez wstawianie).
        \item Usprawnienia wyboru pivota:
            \begin{itemize}
                \item \textbf{Randomizacja} - wszystkie możliwe wielkości
                    podzadań są jednakowo prawdopodobne.
                \item \textbf{Mediana trzech elementów} - wybierz środkowy
                    element (medianę) spośród trzech elementów
                    \texttt{A[L], A[(L+R)/2], A[R]} i zamień go z \texttt{A[R]}
                    (3 porównania więcej). Ta metoda optymalnie radzi sobie z
                    ciągiem uporządkowanym, ale nadal można skonstruować
                    dowolnie długie ciągi, które wymagają czasu $\Omega(n^2)$.
                    Oczekiwana liczba porównań w tym wariancie wynosi około
                    $1.2 n \log_2 n$, ale w praktyce czas wykonania zwykle jest
                    gorszy.
            \end{itemize}
        \item Można porównywać rozmiary podzadań otrzymywanych w wyniku\\
            \texttt{partition()} i iteracyjnie przechodzić do mniejszego, a
            większe zapamiętywać na stosie. Gwarantuje to wielkość stosu
            $O(\log_2 n)$.
        \item Można w ogóle pozbyć się stosu, organizując go wewnętrznie w
            sortowanej tablicy. Zmniejsza się w ten sposób złożoność pamięciową
            do $O(1)$, ale zwiększa się współczynnik proporcjonalności
            złożoności czasowej.
    \end{itemize}

\end{document}
