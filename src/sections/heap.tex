\documentclass[algorytmy.tex]{subfiles}

\begin{document}

\subsection{Definicja}
    Jest to drzewo binarne, w którym:
    \begin{itemize}
        \item dla każdego węzła zachodzi \textbf{warunek kopca}
        \begin{itemize}
            \item \texttt{key(v.parent) >= key(v)}  ~~~~\textit{(max-kopiec)}
        \end{itemize}
        \item wszystkie poziomy, za wyjątkiem ostatniego, są całkowicie
              wypełnione
        \item ostatni poziom jest wypełniony z lewej strony
\end{itemize}

\subsection{Implementacja}
    Postawową implementacją jest \textbf{tablica}.
    \begin{itemize}
        \item \textbf{Lewym} potomkiem węzła \texttt{a[i]} jest
              \texttt{a[2 * i + 1]}
        \item \textbf{Prawym} potomkiem węzła \texttt{a[i]} jest
              \texttt{a[2 * i + 2]}
        \item \textbf{Przodkiem} węzła \texttt{a[i]} jest \texttt{a[(i - 1) / 2]}
    \end{itemize}

% \pagebreak

\subsection{Sortowanie przez kopcowanie}

        \textbf{ZŁOŻONOŚĆ:  } $\Theta(n \log{n})$
\begin{minted}[mathescape,
   linenos,
   numbersep=5pt,
   gobble=2,
   frame=lines,
   % fontsize=\footnotesize,
   framesep=2mm]{java}
    void heapsort(int n) {
        // tworzenie kopca w tablicy a
        // rozpoczynamy analizę od ostatniego elementu,
        // który ma dzieci
        for (int k = (n - 2) / 2; k >= 0; k--)
            downheap(k, n);

        //usuwanie z kopca
        while (n > 0) {
            swap(a[0], a[--n]);
            downheap(0, n);
        }
    }
\end{minted}

\subsection{Przesiewanie w górę}
\begin{minted}[mathescape,
   linenos,
   numbersep=5pt,
   gobble=2,
   frame=lines,
   fontsize=\footnotesize,
   framesep=2mm]{java}
    // podobnie jak podczas sortowania przez wstawianie
    // listą jest ścieżką od węzła k do korzenia
    void upheap(int k) {
        int i = (k - 1) / 2;  // indeks przodka elementu a[k]
        int tmp = a[k];

        while (k > 0 && a[i] < tmp) {
            a[k] = a[i];
            k = i;           // przenieść węzeł w dół
            i = (i - 1) / 2; // przejdź do przodka
        }
       // teraz element a[k] na swoje miejsce
       a[k] = tmp;
    }
\end{minted}

\subsection{Przesiewanie w dół}
\begin{minted}[mathescape,
   linenos,
   numbersep=5pt,
   gobble=2,
   frame=lines,
   fontsize=\footnotesize,
   framesep=2mm]{java}
    // podobnie jak wstawianie do listy w insertsort
    // lista (ścieżka do liścia) wyznaczana dynamicznie
    void downheap(int k, int n) {
        int j = 0;
        int tmp = a[k];

        while (k < n / 2) {
            j = 2 * k + 1; // indeks lewego potomka a[k]

            // wybierz większy z potomków
            if (j < n - 1 && a[j] < a[j + 1])
                j++;

            if (tmp >= a[j])
                break; // warunek kopca OK

            // w przeciwnym wypadku
            // przesuń aktualny element do góry
            a[k] = a[j];
            k = j;
        }
       // teraz element a[k] na swoje miejsce
       a[k] = tmp;
    }
\end{minted}

\subsection{Kolejka priorytetowa}
    \subsubsection{Wstawianie do kolejki}
        \textbf{ZŁOŻONOŚĆ:  } $\Theta(\log{n})$
\begin{minted}[mathescape,
   linenos,
   numbersep=5pt,
   gobble=2,
   frame=lines,
   % fontsize=\footnotesize,
   framesep=2mm]{java}
    boolean insert(int x, int n) {
        if (n == size)
            return false;

        a[n] = x;
        upheap(n++);

        return true;
    }
\end{minted}

    \subsubsection{Odczyt elementu największego}
        \textbf{ZŁOŻONOŚĆ:  } $\Theta(1)$
\begin{minted}[mathescape,
   linenos,
   numbersep=5pt,
   gobble=2,
   frame=lines,
   % fontsize=\footnotesize,
   framesep=2mm]{java}
    int getMax() {
        return a[0];
    }
\end{minted}

    \subsubsection{Usuwanie elementu największego}
        \textbf{ZŁOŻONOŚĆ:  } $\Theta(\log{n})$
\begin{minted}[mathescape,
   linenos,
   numbersep=5pt,
   gobble=2,
   frame=lines,
   % fontsize=\footnotesize,
   framesep=2mm]{java}
    int deleteMax(int n) {
        // zakładamy, że kopiec ma n elementów (n > 0)
        // wstaw ostatni liść w miejsce korzenia
        // i przesiej w dół
        int root = a[0];
        a[0] = a[--n];
        downheap(0, n);
        return root;
    }
\end{minted}



\end{document}
