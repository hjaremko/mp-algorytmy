\documentclass[../algorytmy.tex]{subfiles}

\begin{document}

    \begin{minted}[mathescape,
               linenos,
               numbersep=5pt,
               gobble=2,
               frame=lines,
               framesep=2mm]{java}
    int beginMax = 0;
    int endMax = 0;
    int sumMax = 0;
    int beginBest = 0;
    int currentSum = 0;

    for (int i = 0; i < n; i++) {
        currentSum += arr[i];

        if (currentSum < 0) {
            currentSum = 0;
            beginBest = i + 1;
        }
        else if (currentSum > sumMax) {
            sumMax = currentSum;
            beginMax = beginBest;
            endMax = i;
        }
    }
    \end{minted}

    \textbf{ZŁOŻONOŚĆ}
    \begin{itemize}
        \item \textbf{Pesymistyczna:} ~~$\Theta(n)$
        \item \textbf{Pamięciowa:} ~~$\Theta(1)$
    \end{itemize}

    \textbf{OPIS}
    \begin{itemize}
        \item Metoda oblicza maksymalną podtablicę kończącą się w \texttt{i},
            mając obliczoną maksymalną podtablicę kończącą się w \texttt{i - 1}.
        \item Zauważamy, że maksymalna podtablica dla \texttt{arr[0..i]} jest:
            \begin{itemize}
                \item albo zawarta w \texttt{arr[0..i - 1]}
                \item albo kończy się na \texttt{arr[i]}
            \end{itemize}
    \end{itemize}

    \pagebreak

\end{document}
