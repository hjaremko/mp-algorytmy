\documentclass[../algorytmy.tex]{subfiles}

\begin{document}
    % Przekształcenie algorytmu rekurencyjnego często bazuje na wykorzystaniu
    % stosu i polega na symulacji działania metod rekurencyjnych.

    Po wywołaniu metody jej rekord aktywacji reprezentujący aktualny stan
    wywoływanej funkcji i zawierający:
    \begin{itemize}
        \item obiekty lokalne – zmienne, stałe, parametry
        \item adres powrotu (aktualny licznik rozkazów)
    \end{itemize}
    jest wstawiany na stos i następuje skok do początku kodu metody.
    Bezpośrednio przed zakończeniem działania metody ze stosu pobierany jest
    rekord aktywacji, a następnie są odtwarzane obiekty lokalne z przed
    wywołania i następuje skok do adresu powrotu w pobranym rekordzie.

    \pagebreak
\end{document}
