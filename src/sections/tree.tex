\documentclass[algorytmy.tex]{subfiles}

\begin{document}
    \subsection{Preorder}
        \textbf{REKURENCYJNIE}
\begin{minted}[mathescape,
   linenos,
   numbersep=5pt,
   gobble=2,
   frame=lines,
   framesep=2mm]{java}
    void preorder(Node root) {
        if (root != null) {
            print(root.info);
            preorder(root.left);
            preorder(root.right);
        }
    }
\end{minted}

        \textbf{ITERACYJNIE}
\begin{minted}[mathescape,
   linenos,
   numbersep=5pt,
   gobble=2,
   frame=lines,
   framesep=2mm]{java}
    void preorder() {
        Stack<Node> stack = new Stack<>();
        Node current = root;

        while (current != null || !stack.isEmpty()) {
            if (current != null) {
                print(current.info);
                stack.push(current.right);
                current = current.left;
            }
            else {
                current = stack.pop();
            }
        }
    }
\end{minted}

\pagebreak

\subsection{Inorder}
        \textbf{REKURENCYJNIE}
\begin{minted}[mathescape,
   linenos,
   numbersep=5pt,
   gobble=2,
   frame=lines,
   framesep=2mm]{java}
    void inorder(Node root) {
        if (root != null) {
            inorder(root.left);
            print(root.info);
            inorder(root.right);
        }
    }
\end{minted}

        \textbf{ITERACYJNIE}
\begin{minted}[mathescape,
   linenos,
   numbersep=5pt,
   gobble=2,
   frame=lines,
   framesep=2mm]{java}
    void inorder() {
        Stack<Node> stack = new Stack<>();
        Node current = root;

        while (current != null || !stack.isEmpty()) {
            if (current != null) {
                stack.push(current);
                current = current.left;
            }
            else {
                current = stack.pop();
                print(current.info);
                current = current.right;
            }
        }
    }
\end{minted}

\pagebreak

\subsection{Postorder}
        \textbf{REKURENCYJNIE}\\
\begin{minted}[mathescape,
   linenos,
   numbersep=5pt,
   gobble=2,
   frame=lines,
   framesep=2mm]{java}
    void postorder(Node root) {
        if (root != null) {
            postorder(root.left);
            postorder(root.right);
            print(root.info);
        }
    }
\end{minted}

        \textbf{ITERACYJNIE}\\
\begin{minted}[mathescape,
   linenos,
   numbersep=5pt,
   gobble=2,
   frame=lines,
   fontsize=\footnotesize,
   framesep=2mm]{java}
    void postorder() {
        Stack<Node> stack = new Stack<>();
        Node current = root;

        while (current != null || !stack.isEmpty()) {
            while (current != null) {
                if (current.right != null) {
                    stack.push(current.right);
                }

                stack.push(current);
                current = current.left;
            }

            current = stack.pop();

            if (current.right != null && stack.top() == current.right) {
                stack.pop();
                stack.push(current);
                current = current.right;
            }
            else {
                print(current.info);
                current = null;
            }
        }
    }
\end{minted}

\pagebreak

\subsection{Levelorder}

\begin{minted}[mathescape,
   linenos,
   numbersep=5pt,
   gobble=2,
   frame=lines,
   framesep=2mm]{java}
    void levelorder() {
        if (root == null) {
            return;
        }

        Queue<Node> queue = new Queue<>();
        queue.pushBack(root);

        while (!queue.isEmpty()) {
            Node current = queue.popFront();
            print(current.info);

            if (current.left != null) {
                queue.pushBack(current.left);
            }

            if (current.right != null) {
                queue.pushBack(current.right);
            }
        }
}
\end{minted}


\subsection{Drzewa poszukiwań binarnych (BST)}

Jest to drzewo binarne, w którym lewe poddrzewo każdego węzła zawiera
\textbf{wyłącznie elementy o kluczach nie większych niż klucz węzła a prawe
poddrzewo zawiera wyłącznie elementy o kluczach nie mniejszych niż klucz węzła}.\\

Dla pełnego drzewa BST o $n$ węzłach pesymistyczny koszt każdej z podstawowych
operacji wynosi $O(\log_2 n)$.\\

Drzewo binarne jest BST wtedy i tylko wtedy gdy lista jego węzłów w porządku
\texttt{inorder} jest ciągiem \textbf{niemalejącym}.\\

\subsubsection{Wyszukanie wartości minimalnej}

\begin{minted}[mathescape,
   linenos,
   numbersep=5pt,
   gobble=2,
   frame=lines,
   framesep=2mm]{java}
    Node minimum() { // Maksymalna analogicznie tylko w prawo
        Node current = root;
        Node last;

        while (current != null) {
            last = current;
            current = current.left;
        }

        return last;
    }
\end{minted}

\subsubsection{Wyszukanie danego klucza}

\begin{minted}[mathescape,
   linenos,
   numbersep=5pt,
   gobble=2,
   frame=lines,
   framesep=2mm]{java}
    Node search(int key) {
        Node current = root;

        while (current != null && key != current.info) {
            if (key < current.info) {
                current = current.left;
            }
            else {
                current = current.right;
            }
        }

        return current;
    }
\end{minted}

\pagebreak

\subsubsection{Wstawianie danego klucza}
\textbf{ITERACYJNIE}
\begin{minted}[mathescape,
   linenos,
   numbersep=5pt,
   gobble=2,
   frame=lines,
   fontsize=\footnotesize,
   framesep=2mm]{java}
    void insert(int key) {
        Node s, p, prev;
        s = new Node(key); //s.left = null; s.right = null;

        if (root == null) //drzewo puste
            root = s;
        else {
            p = root;
            prev = null;

            while (p != null) {
                prev = p;

                if (key < p.info)
                    p = p.left;
                else
                    p = p.right;
            }

            if (key < prev.info)
                prev.left = s;
            else
                prev.right = s;
        }
    }
\end{minted}

% \pagebreak

\textbf{REKURENCYJNIE}
\begin{minted}[mathescape,
   linenos,
   numbersep=5pt,
   gobble=2,
   frame=lines,
   fontsize=\footnotesize,
   framesep=2mm]{java}
    void insert(Node p, int key) { //wywołanie: insert(root, x);
        if (p == null) { //baza rekurencji, tworzenie węzła
            p = new Node(key);

            if (root == null)
                root = p;
        }
        else {
            if (key < p.info)
                p.left = insert(p.left, key);
            else
                p.right = insert(p.right, key);
        }

        return p; //referencja do wstawianego węzła
    }
\end{minted}

\pagebreak

\subsubsection{Usuwanie danego klucza}

\textbf{ITERACYJNIE}
\begin{minted}[mathescape,
   linenos,
   numbersep=5pt,
   gobble=2,
   frame=lines,
   fontsize=\footnotesize,
   framesep=2mm]{java}
    void delete(int key) {
        Node parent = _getParent(key, root);
        Node _root = root;

        while (true) {
            parent = _getParent(key, parent);
            Node curr = find(key, _root);

            if (curr == null) return;

            //oba poddrzewa puste
            if (curr.left == null && curr.right == null) {
                if (curr != root) {
                    if (parent.left == curr)
                        parent.left = null;
                    else
                        parent.right = null;
                }
                else root = null;

                return;
            } //oba poddrzewa niepuste
            else if (curr.left != null && curr.right != null) {
                Node succ = min(curr.right);
                curr.info = succ.info;

                key = succ.info;
                _root = curr.right;
                parent = curr;
            } //jedno niepuste poddrzewo
            else {
                Node child = (curr.left != null) ? curr.left : curr.right;

                if (curr != root) {
                    if (parent.left == curr)
                        parent.left = child;
                    else
                        parent.right = child;
                }
                else root = child;

                return;
            }
        }
    }
\end{minted}

\textbf{REKURENCYJNIE}\\
\begin{minted}[mathescape,
   linenos,
   numbersep=5pt,
   gobble=2,
   frame=lines,
   % fontsize=\footnotesize,
   framesep=2mm]{java}
    void delete(int key, Node root) {
        // baza
        if (root == null)
            return root;

        if (key < root.info)
            root.left = delete(key, root.left);
        else if (key > root.info)
            root.left = delete(key, root.right);
        else {// klucz taki sam jak roota wiec usuwamy go
            if (root.left == null)
                return root.right;
            else if (root.right == null)
                return root.left;

            // węzeł z dwoma potomkami
            // weź następnik (najmniejszy w prawym)
            root.info = minValue(root.right);

            // usuń następnik
            root.right = delete(root.info, root.right);
        }

        return root;
    }
\end{minted}

\pagebreak

% \subsubsection{Wyszukanie danego klucza}
% \begin{minted}[mathescape,
%    linenos,
%    numbersep=5pt,
%    gobble=2,
%    frame=lines,
%    fontsize=\footnotesize,
%    framesep=2mm]{java}
%     void search(int key) {
%         Node p = root; //zaczynamy od korzenia

%         if (p == null) return null;

%         while (p.info != key) { //dopóki nie znaleziono
%             if (key < p.info)
%                 p = p.left;
%             else
%                 p = p.right;

%             if (p == null) //brak potomka -> nie odnaleziono
%                 return null;
%         }

%         return p; //odnalzeiono
%     }
% \end{minted}

\subsubsection{Wyszukanie rodzica danego klucza}
\begin{minted}[mathescape,
   linenos,
   numbersep=5pt,
   gobble=2,
   frame=lines,
   fontsize=\footnotesize,
   framesep=2mm]{java}
    void parent(int key) {
        Node p = root; //zaczynamy od korzenia
        Node prev = null;

        if (p == null)
            return null;

        if (p.info == key)
            return null;

        while (p != null && p.info != key) { //dopóki nie znaleziono
            prev = p;

            if (key < p.info)
                p = p.left;
            else
                p = p.right;

            if (p == null) //brak potomka -> nie odnaleziono
                return null;
        }

        return prev; //odnalzeiono
    }
\end{minted}

\subsubsection{Wyszukanie poprzednika danego klucza}
\begin{minted}[mathescape,
   linenos,
   numbersep=5pt,
   gobble=2,
   frame=lines,
   fontsize=\footnotesize,
   framesep=2mm]{java}
    void predecessor(int key) {
        Node p, q;
        q = search(key);

        if (q == null) //nie ma klucza zwraca null
            return null;

        if (q.left != null) { //szukamy maksimum w lewym przedziale
            p = q.left;

            while (p.right != null)
                p = p.right;

            return p;
        }

        p = parent(key);

        while (p != null && q == p.left) {
            q = p;              //idziemy w górę i szukamy węzeł p
            p = parent(p.info); //którego prawym następnikiem jest q
        }

        return p; //zwraca null jeśli nie ma poprzednika
    }
\end{minted}

\subsubsection{Wyszukanie następnika danego klucza}
\begin{minted}[mathescape,
   linenos,
   numbersep=5pt,
   gobble=2,
   frame=lines,
   fontsize=\footnotesize,
   framesep=2mm]{java}
    void successor(int key) {
        Node p, q;
        q = search(key);

        if (q == null) //nie ma klucza zwraca null
            return null;

        if (q.right != null) { //szukamy minimum w prawym przedziale
            p = q.right;

            while (p.left != null)
                p = p.left;

            return p;
        }

        p = parent(key);

        while (p != null && q == p.right) {
            q = p;              //idziemy w górę i szukamy węzeł p
            p = parent(p.info); //którego prawym następnikiem jest q
        }

        return p; //zwraca null jeśli nie ma następnika
    }
\end{minted}

\pagebreak

\end{document}
