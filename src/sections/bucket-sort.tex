\documentclass[advanced-sorts.tex]{subfiles}

\begin{document}
    \textbf{ZAŁOŻENIA}
    \begin{itemize}
        \item Ciąg kluczy do posortowania to
            \texttt{a[0], a[1], ..., a[n - 1]} \\ typu
            \underline{rzeczywistego}.
        \item Wartość każdego klucza należy do uporządkowanego zbioru np. liczb
            rzeczywistych: $\{ q_1 < q_2 < \ldots < q_m \}$, to znaczy\\
            $ \forall ~~ i \in \{ 0, \ldots, n - 1 \} ~~~ \exists ~~ j \in
            \{ 1, \ldots, m \} : \mathtt{a[i]} = q_j $
    \end{itemize}

    \textbf{OPIS METODY}
    \begin{enumerate}
        \item Utwórz $m$ kubełków (np. list), początkowo pustych.
        \item Dla \texttt{ i = 0, ..., n - 1 }, jeśli \texttt{a[i] = }$q_j$
            to wstaw \texttt{a[i]} do kubełka o numerze $j$
            \textit{(wyszukiwanie binarne)}.
        \item Uczyń tablicę \texttt{a[]} pustą.
        \item Kolejno dla \texttt{ j = 1, ..., m } przepisz zawartość kubełka $j$
            do tablicy \texttt{a[]} (tak więc w \texttt{a[]} pojawią się
            najpierw klucze o wartości $q_1$, potem klucze o wartości $q_2$,
            itd.).
    \end{enumerate}

    \textbf{PSEUDOKOD}
    \begin{minted}[mathescape,
               linenos,
               numbersep=5pt,
               gobble=2,
               frame=lines,
               fontsize=\footnotesize,
               framesep=2mm]{java}
    double Q[m];  // Q = {q1, q2, ..., qm}
    int count[m]; // tablica liczników

    for (int j = 0; j < m; j++)
        count[j] = 0;

    for (int i = 0; i < n; i++) {
        //m-numer kubełka do którego należy a[i]
        int m = binary_search(a[i], Q);
        count[m]++;
    }

    for (int j = 0; j < m; j++) {
        for (int p = 0; p < count[j]; p++)
            wypisz(Q[j]);
    }
   \end{minted}

    \textbf{ZŁOŻONOŚĆ}
    \begin{itemize}
        \item \textbf{Czasowa:} ~~$O(n \log_2 m)$
        \item \textbf{Pamięciowa:} ~~$O(n + m)$
    \end{itemize}

    % \textbf{ZALETY}
    % \begin{itemize}
    %     \item .
    % \end{itemize}

    % \textbf{WADY}
    % \begin{itemize}
    %     \item .
    % \end{itemize}

    % \textbf{MOŻLIWE USPRAWNIENIA}
    % \begin{itemize}
    %     \item .
    % \end{itemize}

\end{document}
