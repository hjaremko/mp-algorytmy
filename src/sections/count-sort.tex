\documentclass[advanced-sorts.tex]{subfiles}

\begin{document}
    \textbf{ZAŁOŻENIA}
    \begin{itemize}
        \item Elementy tablicy
            \texttt{a[0], a[1], ..., a[n - 1]} przyjmują\\
            \underline{wartości nieujemne i mniejsze niż \textbf{m}}, to znaczy
            dla\\ $i = 0, \ldots, n - 1  ~~~ \mathtt{a[i]} \in \{ 0, 1, \ldots, m - 1 \}$
        \item Tablice pomocnicze: \texttt{b[n], count[m]}
    \end{itemize}

    \textbf{PSEUDOKOD}
    \begin{minted}[mathescape,
               linenos,
               numbersep=5pt,
               gobble=2,
               frame=lines,
               fontsize=\footnotesize,
               framesep=2mm]{java}
    for (int j = 0; j < m; j++) count[j] = 0;

    for (int i = 0; i < n; i++)
        count[a[i]]++;

    for (int j = 1; j < m; j++)
        count[j] += count[j - 1];

    for (int i = n - 1; i >= 0; i--)
        b[--count[a[i]]] = a[i];

    for (int i = 0; i < n; i++) a[i] = b[i];
    \end{minted}

    \textbf{OPIS METODY}
    \begin{enumerate}
        \item Zerowanie tablicy \texttt{count[]}.
        \item Zliczanie - ile jest elementów w tablicy \texttt{a[]} o danej
            wartości.
        \item Obliczanie górnych granic obszarów wynikowych dla poszczególnych
            wartości.
        \item \underline{Przepisz od końca} \texttt{a[]} do \texttt{b[]} na
            właściwe miejsca \textit{(zapewnia \textbf{stabilność})}.
        \item Opcjonalnie, mozna przepisać \texttt{b[]} z powrotem do
            \texttt{a[]}.
    \end{enumerate}

    \textbf{ZŁOŻONOŚĆ}
    \begin{itemize}
        \item \textbf{Czasowa:} ~~$\Theta(n + m)$
        % \item \textbf{Pamięciowa:} ~~$O(n + m)$
    \end{itemize}

    % \textbf{ZALETY}
    % \begin{itemize}
    %     \item \textbf{Stabilność.}
    % \end{itemize}

    \textbf{WADY}
    \begin{itemize}
        \item Pamięć pomocnicza wielkości $\Theta(n + m)$.
        \item Wartości elementów muszą być liczbami całkowitymi ograniczonej
            wielkości.
    \end{itemize}

    % \textbf{MOŻLIWE USPRAWNIENIA}
    % \begin{itemize}
    %     \item .
    % \end{itemize}

\end{document}
