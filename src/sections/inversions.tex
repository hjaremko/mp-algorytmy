\documentclass[../algorytmy.tex]{subfiles}

\begin{document}

    \begin{minted}[mathescape,
               linenos,
               numbersep=5pt,
               gobble=2,
               frame=lines,
               fontsize=\footnotesize,
               framesep=2mm]{java}
    long numInversion(int left, int right) //mergeSort
    { // wywołanie: numInversion(0, size - 1)
        long invs = 0;

        if (right > left) {
            int mid = (left + right) / 2;

            invs  = numInversion(left, mid); // ilosc z lewej i prawej czesci
            invs += numInversion(mid + 1, right);
            invs += merge(left, mid, right); // ilosc inwersji z laczenia
        }

        return invs;
    }

    long merge(int left, int mid, int right)
    {
        copyToTemp(from left to mid); //kopiujemy tylko połowę

        int i = left;
        int j = mid + 1;
        int k = left;

        long invs = 0;

        while (i <= mid && j <= right) {
            if (temp[i] <= data[j])
                data[k++] = temp[i++];
            else {//temp[i] > data[j]
                data[k++] = data[j++];

                invs += mid + 1 - i; //skoro lewa i prawa polowa sa
                // posortowane to pozostale elementy w lewej polowie
                // sa wieksze od data[j], zatem jest mid + 1 - i inwersji
            }
        }

        while (i <= mid) //przepisz pozostale
            data[k++] = temp[i++];

        while (j <= right)
            data[k++] = data[j++];

        return invs;
    }
    \end{minted}

\end{document}
